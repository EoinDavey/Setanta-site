\chapter{Background}

\section{Executing Code on the Browser}

The only PL that standard web browsers (\emph{Google Chrome, Mozilla Firefox, Safari, Edge, Internet Explorer, etc.}) can execute is JavaScript. There is a movement to bring other PLs to the browser via a technology called WebAssembly, but this has not fully come to fruition yet.  

In addition to this restriction on PLs, each browser instance limits executing code to a single, non-blocking thread.
To overcome the non-blocking limitation of the execution thread the JavaScript environment exposes an API to a simple task queue, allowing users to pass in \textbf{callbacks}, which are functions that will be executed at some later time. In Figure \ref{blockingcode} we compare code in a version of JS that allows blocking, and show the real code that must be used using callbacks using the \texttt{setTimeout} API to add the callback to the task queue.

\begin{figure}[ht]
    \caption{Code if thread can be blocked vs real situation.}
    \label{blockingcode}
    \begin{minipage}[t]{0.45\textwidth}
        \begin{lstlisting}[language=javascript]
// Would work in an ideal world
function sleepFor100ms() {
    print('Starting sleep')
    sleep(100)
    print('After sleep')
}
        \end{lstlisting}
    \end{minipage}\qquad
    \begin{minipage}[t]{0.45\textwidth}
        \begin{lstlisting}[language=javascript]
// The real way to achieve this
function sleepFor100ms() {
    print('Starting sleep')
    setTimeout(() => print('After sleep'), 100)
}
        \end{lstlisting}
    \end{minipage}
\end{figure}

\section{TypeScript}
TypeScript is an open source PL being developed by Microsoft, TypeScript is the primary PL that is used in the implementation of this project. Microsoft describes it as

\begin{quote}
TypeScript is a typed superset of JavaScript that compiles to plain JavaScript.\cite{microsoftts}
\end{quote}

TypeScript allows developers to write JavaScript programs with all the benefits of compile time static type checking.

\section{PEG and CFG Grammars}

Formal grammars are sets of rules for re-writing strings, first formally researched and defined by Noam Chomsky, 1956\cite{chomskypaper}. In recent times they find frequent use in specifying the syntactic grammar of various PLs.

The formal grammar family that has found the most usage is that of CFGs, Context Free Grammars. PEG Grammars are a new type of grammar that has come into prevalent usage recently, they are very similar to CFG grammars but are defined in such a way that they cannot be ambiguous. It is conjectured that PEGs are more powerful than CFGs\cite{pegconjecture}

\section{Parser Generators}

A parser generator is a program that, given a description of a grammar in some formal syntax, outputs a parser for that grammar. Parser generators have existed since the earliest days of computer programming. One of the earliest major parser generators, YACC, was released in 1975 \cite{yacc}, and was based on LR parsing.

Several parser generators are available for JavaScript, including the excellent \emph{pegjs} generator. However, no such parser generator was available for TypeScript, this is what prompted the creation of \tsPEG{}.

\section{VSO vs SVO languages}\label{vsosvo}

In linguistics the order of how the different categories of words appear in sentences is often studied. One order that is studied is called the ``constituent word order''. This is the order that a verb (V), object (O), and subject (S) appear in a sentence\cite{wordorder}.
English is an SVO language, meaning that the subject comes first, then the verb, then the object. Irish is a VSO language, meaning the verb comes first, then the subject, then the object.
