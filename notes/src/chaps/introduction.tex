\chapter{Introduction}
For easier distinction between programming languages and "human" languages, from this point I will refer to programming languages as \textbf{PLs}, and traditional languages as just \textbf{languages}.
\section{Motivation}
English is the language of choice for the programming world, even PLs developed in non English speaking countries are designed to be written in English, e.g. \emph{Lua} (developed in the Netherlands), \emph{Ruby} (developed in Japan). This focus on one single language must have some impact the way we design our PLs. Many PLs have been written for other languages, but if you go to use one you will almost certainly find that it is a \emph{translation} of a PL originally written for English speakers\cite{wikipllist}. If we design a PL from the ground up around a non English language, what changes do we see between it and the industry standard English PLs.

Irish was chosen as the language to build the new PL around for many reasons, the obvious being that it is the native language of Ireland, so it is of interest to an Irish audience, but this is not the only reason. Ireland is a language that historically has faced significant hostility, and today finds itself a minority language in its own country, however it is still spoken by over 73,000 people daily\cite{csoirish}. If any Irish speaking person wishes to learn about programming, they have no choice but to do it through the medium of English. By creating an Irish PL and an online learning environment around it I hope to enable people to learn to program in they way that they want to.
\section{Problem Statement}
This project involves the design and implementation of a new, modern, innovative PL, entirely in Irish. This PL will not be a translation or a modification of a previously existing PL, this is to allow the PL design to be influenced by the Irish language at every stage.

The PL should be designed with education in mind. It will be built to run in the browser, in order to enable high ease of access to as many people as possible. By running the code in the browser, no installations are required to use the PL, just a web browser.

The PL must be a modern PL with all industry standard features, this is to ensure that by learning the PL, you learn the most fundamental programming concepts.

An online learning environment will be created where the user can write and execute the PL in the browser. It should be accessible and easy to use. To assist in the learning process the environment will take inspiration from popular educational tools like \emph{Scratch} and \emph{Logo} and have a graphical interface where the user can draw shapes and interact with visual elements. Research has shown that the use of visual elements in educational approaches improves the learning experience\cite{graphiclearning}.

To implement an interpreter for a PL a parser is needed, usually a parser generator is used to do this. However, as not many languages are built to be browser-first, the parser generator choices available for TypeScript (my PL of choice for this project) were not quite suitable. This leads to the additional part of this project to create a novel parser generator for TypeScript. The parser generator must be powerful enough to support the Irish PL, as well as to be capable of bootstrapping its own parser. It should be built on the latest innovations in parsing technology, providing accurate syntax error detection and ASTs to the user. The ASTs generated by the parser should be strongly typed, to enable maximum utility of the TypeScript type system.
\section{Approach}
\section{Metrics}
\section{Project}
