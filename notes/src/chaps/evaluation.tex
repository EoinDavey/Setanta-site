\chapter{Evaluation}

In this chapter I will address the evaluation of each of the main components independently.

\section{\tsPEG{}}

The \tsPEG{} parser generator was implemented almost precisely to the original design. The correctness and effectiveness of \tsPEG{} can be seen in two direct ways.
\begin{itemize}
    \item \tsPEG{} is self-hosting, meaning that the input parser for \tsPEG{} is generated by \tsPEG{} itself. This would not be possible if \tsPEG{}s parsers were not correct. The input grammar makes use of all the main features of \tsPEG{}, so this acts as a rigorous self-referential test, whereby the whole project fails to even compile if it is not correct.
    \item \tsPEG{} generates the parser for the \Setanta{} interpreter. This proves that \tsPEG{} is capable of generating parsers for fully featured programming languages, with complex and intricate grammars. The use of \tsPEG{} in the \Setanta{} build process serves as a full stress test of all the features available to it.
\end{itemize}

\tsPEG{} has also attracted attention from the TypeScript community, showing that it serves a useful purpose as a tool for the community. However \tsPEG{} still has many features that it would benefit hugely from, the main one being improving the syntax error reporting capabilities, and efficiency.

\section{\Setanta{}}

As outlined in the solution section of \Setanta{} (\ref{setantasolution}), the implementation of \Setanta{} was a success, it meets every goal that was set out originally for it.

The correctness of \Setanta{} is ensured by a large suite of tests, including unit tests, and full end to end tests. In fact \Setanta{} contains over 1300 lines of test code alone. These tests ensure that every edge of language behaviour is correct, theses tests are run automatically by the continuous integration system with each commit. The current state of \Setanta{}'s builds can be seen at \href{https://travis-ci.com/EoinDavey/Setanta}{travis-ci.com/EoinDavey/Setanta.}

\Setanta{}'s main issue is it's speed. As it is built on top of the JavaScript environment, but requires blocking operations and concurrency, it means that \Setanta{} has taken on an extremely large overhead. The result of this is that that \Setanta{} is notably slow at some tasks. However, the domain of \Setanta{} is education, not high performance or production computing, so this is not a direct detractor from the quality of the project.

The fact that the same \Setanta{} interpreter can be used in the browser and in a local CLI shows the code is well abstracted and isolated, the main \lstinline|Interpreter| class is very portable and can loaded in any number of environments.

\Setanta{} has received a notable amount of attention online, before I had even announced the project I had received requests to use the project from people who had stumbled upon the repository. I gave a talk on \Setanta{} at SISTEM 2020 (A tech conference held in UCD) and it received high praise from many of the attendees.

\section{\trys{}}

The \trys{} learning environment has also been largely successful, and meets most of the original design goals for it. \trys{} is a simple to use website, where the user can write and execute \Setanta{} code, it has a good editor with syntax highlighting and a powerful graphics API.
The implementation of \trys{} is quite modern, taking advantage of the latest features of HTML5 and CSS3. \trys{} is built on the web components technology, this means that the main components of the \trys{} website are fully isolated, abstracted and re-usable.

\trys{} doesn't meet the original design plan of having an extremely high level graphics API, as I found in the development process of this API that it was simpler to understand a simple API that let you call a function to draw circles, squares etc. than a very high level API where you would need to use classes and inheritance and other features to interact with the API.

\section{Overall end to end correctness}

The act of going to \trys{}, writing a game in \Setanta{}, executing it, and playing the game in the browser is a full end to end proof that each component, \trys{}, \Setanta{} and \tsPEG{} are all fully functional, as each part is 100\% vital to the project as a whole.
