\documentclass[11pt]{extarticle}
\usepackage[a4paper, margin=0.75in]{geometry}
\newcommand{\Setanta}{\emph{Setanta}}
\begin{document}
    \title{
    \huge \Setanta{} - An Irish programming language and learning environment\\
    \Large Interim Report\\
    \large CS460 Final Year Project\\
    2019 - 2020}
    \author{\Large Student: Eoin Davey - 16334926 \and Supervisor: Dr. Barak Pearlmutter}
    \maketitle
    \section*{Goals of the project}

    The primary goal of this project is to create a new, modern programming language built around the Irish language, \Setanta{}, with an intuitive online learning environment, allowing the language to be used as an educational tool, with no installation required.

    \Setanta{} is a modern, dynamic, object oriented language, whose domain is education. It's easily accessible and intuitive, it will feel familiar to anyone with some programming experience, while being friendly enough to learn as a first language. \Setanta{} takes inspiration from the Irish language and culture, for simple things like it's keywords, down to it's deeper semantics.

    The \Setanta{} learning environment provides an intuitive interface to program \Setanta{} with. It provides a modern text editor, an I/O console, as well as a powerful API to interact with a custom graphics environment, all running in the browser.

    The project also involves development of new tools to help in programming language implementation for the web, namely building an expressive, powerful parser generator for TypeScript. This parser generator is then used to develop the interpreter for \Setanta{}.

    \section*{Background}

    English is well and truly the lingua franca of the programming world, even languages developed outside the Anglosphere are designed to be written in English, for example \emph{Ruby} is from Japan and \emph{Lua} is from Brazil, but both are designed around the English language. Part of the aim of this project is to explore what syntactic and semantic constructs that a language not designed around English might have.

    The Irish language is in daily use around Ireland, it is a strong living language in many areas, and has a strong community of enthusiastic speakers. It is the primary language in many schools, both primary and secondary, around the country. Ireland's education system is beginning to bring more and more Computer Science education into the school curriculum.
    
    For this reason and many others, there will be more and more Irish speaking people looking to get started learning programming, and what better way than by learning on an accessible platform, in a modern language, and in their own native tongue.
    \section*{Progress to Date}
    \section*{Problems Encountered}
    \section*{Planned Steps to Completion}
\end{document}
