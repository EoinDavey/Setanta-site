\chapter{\Setanta{} syntax outline}
The syntax of \Setanta{} is new, but should feel familiar to most people. It has been designed to be simple and approachable. It takes inspiration from C like languages, but has some new ideas of it's own.
The grammar has been designed for unambiguity so no semicolons or similar construct are required.

\Setanta{} programs, like most imperative languages, consist of a sequence of statements. Some important \Setanta{} features are outlined below:
\begin{itemize}
    \item \textbf{Variable declarations}

        In \Setanta{} variables are declared using the \verb|:=| operator, and can be re-assigned using the classic \verb|=| operator. The distinction is to provide a clear lexical difference between variable declaration and reassignment.

            \begin{lstlisting}[language=setanta, frame=single, caption=Variables]
x := 0
x = x + 1
            \end{lstlisting}
    \item \textbf{Conditionals}

        \Setanta{} support the classic conditional execution structure of if, else if, else. This is mostly a direct translation into Irish, as it uses the keyword \lstinline[language=setanta]|má| meaning ``if''. However it should be noted that no bracketing is required around the expression.

            \begin{lstlisting}[language=setanta, frame=single, caption=Setanta conditionals]
má x == 0
    scríobh('Tá x cothrom le 0')
nó má x == 1
    scríobh('Tá x cothrom le 1')
nó
    scríobh('Tá x níos mo ná 1')
            \end{lstlisting}
        \item \textbf{Loops}

            \Setanta{} supports two main types of loops, ``le idir'' loops that allow the user to specify start and ends to the loop, and ``nuair-a'' loops, which are the familiar while loops.

            \begin{lstlisting}[language=setanta, frame=single, caption=Setanta loops]
i := 0
le i idir (0, 10)
    i = i + 1
x := 0
nuair-a x < 10
    x = x + 1
            \end{lstlisting}
        \item \textbf{Functions}

            Functions in \Setanta{} are referred to with the term ``gníomh'' meaning ``action''. They can constructed with the \lstinline[language=setanta]|gníomh| keyword, followed by a name and argument list. The \lstinline[language=setanta]|toradh| keyword can be used to return values from the function.
            \begin{lstlisting}[language=setanta, frame=single, caption=Setanta function example]
gníomh fibonacci(n) {
    má n <= 1
        toradh 1
    toradh fibonacci(n-1) + fibonacci(n-2)
}
            \end{lstlisting}
        \item \textbf{Classes}

            \Setanta{} supports declaring new classes, with methods, and a constructor.  Classes can inherit from other classes using the keyword \emph{ó}
            \begin{lstlisting}[language=setanta, frame=single, caption=Setanta classe example]
creatlach Person ó Animal {
    gníomh nua(name) {
        name@seo = name
    }
    gníomh speak() {
        scríobh('Hi, My name is ' + name@seo)
    }
}
            \end{lstlisting}
        \item \textbf{Literals}

            \Setanta{} supports literals for integers, booleans, null, strings and lists.
            \begin{lstlisting}[language=setanta, frame=single, caption=Setanta literals]
a := 500
b := 'Dia duit domhan'
c := [1,2,3,4, fíor]
d := fíor != breag
c := neamhní
            \end{lstlisting}
\end{itemize}
